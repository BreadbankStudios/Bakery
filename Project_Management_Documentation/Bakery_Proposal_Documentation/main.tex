\documentclass[12pt]{article}
\usepackage[utf8]{inputenc}
\usepackage{cite}
\usepackage{lipsum}
\usepackage{hyperref}
\hypersetup{
    colorlinks=true,
    linkcolor=black,
    filecolor=magenta,      
    urlcolor=blue,
    citecolor=black,
}

\title{Bakery | Encrypted Mail Client}
\author{Jim Kelly writing for Breadbank Studios}

\begin{document}
\maketitle
\vspace{40pt}
\section*{Abstract}
	\noindent `Bakery' is a multi-platform desktop application mail client that has been designed and developed to demonstrate the capabilities of the authors, regarding their software development practices and further their understanding of JavaScript, encryption, and user interface \& user experience (UI/UX) design practices. This document houses the referenced projects, project design documentation, and the tools and software that will be used for this project.\\
	\newline
	\noindent In short summary the mail client will follow the basic structure of a mail client by fetching user emails via a given email account, i.e. Google, Outlook, etc, alongside having the capacity to send emails from the account or accounts that the application is provided access to. Additional functionality comes in the form of end-to-end encryption (E2EE) between users sending emails through the client. This is to be achieved by public and private key infrastructure, the details of this process are to be confirmed at a later date.\\
	\newline
	\noindent Regarding the technologies used for this project the primary language to be used in the construction of the application is JavaScript, primarily through React-Native and Electron. This includes potentially `Storybook'\cite{storybooks} for UI component development. Any later additional technologies implemented in the development of this project will be documented here.
\pagebreak
\tableofcontents
\pagebreak
	\section{Introduction}
		Modern day systems have become rapidly more and more involved with the processes of digital communication, with email essentially becoming the de-facto replacement for physical mail systems with the notification of important documents or sensitive information. ``According to the Breach Level Index, over 13 million records have leaked or been lost in published cybersecurity breaches since 2013. Of those 13 million records, a terrifying 96 percent weren't encrypted.''\cite{zapierSecure}. Naturally, the most common method for an attacker to be able to crack security systems and proceed to use phishing attacks to impersonate trusted actors within the ecosystem is through the access to the contents of emails. While the most common email service providers, namely Google Gmail or Microsoft Outlook provide a decent level of protection as a service regarding the security of a user's emails. There is little to no guarantee beyond the security systems in place that someone else is not reading or somehow gaining access to one's email account.\\
		\newline
		\noindent It must be noted that contemporary forms of encryption are so efficient that it would approximately take one million computers, working collectively for sixteen million years\cite{encryptionStrength} to be able to gain access. While this provides much comfort to the standard user, services such as Gmail and Hotmail only act to encrypt user data as it travels from the user's computer to the server. In simplest terms, this means that on both ends of the server, the contents of the email are readily available to be read in plaintext form. Furthermore, this requires trust in the organisation hosting the email service that they will not use their encryption algorithm to decrypt the user's email or disclose the email's contents to unauthorised users. This spruiks the necessity for the development and deployment of a mail-client that achieves seamless end-to-end encryption which gives control to the users. Essentially, the objective of this project is to develop a service whereby users have the option to encrypt their emails sent through the mail-client, using public-key infrastructure. This process is enabled by the construction of a peer network, whereby the public key of the target of the email (be that an organisation, entity, or single user) is used to encrypt the contents of the email. Furthermore, the plaintext of the email is not stored by a server or through the user's given email service provider as it is encrypted locally before being transmitted. It must be noted than a optional feature to store the plaintext contents locally for posterity is provided to the user.
\section{Project Specifications}
	\subsection{Similar Projects in Development}
		There are several projects similar to or alike to the `Bakery' mail client. Two notable similar desktop applications are `ElectronMail'\cite{ElectronMail} and `Mailspring'\cite{Mailspring}.
	\subsection{Electron: Open Source Framework}
		\lipsum[1]
	\subsection{Public \& Private Key Infrastructure}
		\lipsum[1]
\pagebreak
\section{Project Planning}
	\lipsum[1]
	\pagebreak
	\subsection{Work Breakdown Structure (WBS)}
	\lipsum[1]
	\pagebreak
	\subsection{Sequence Diagram}
	\lipsum[1]
	\pagebreak
	\subsection{Component Diagram}
	\lipsum[1]
	\pagebreak
	\subsection{Class Diagram}
	\lipsum[1]
	\pagebreak
	\subsection{Deployment Diagram}
	\lipsum[1]
\pagebreak
\bibliographystyle{IEEEtran}
\bibliography{main}

\end{document}
